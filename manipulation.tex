\begin{enumerate}
    \item{La pompe à main doit être branchée avec le tube ayant son extremité scellée orientée vers le haut}
    \item{Tapoter le tube afin de réunir en une seule goutte les éclats de mercure disposés le long du tube}
    \item{Avec la pompe, générer une mesure de $\Delta p$ maximale}
\end{enumerate}

\begin{enumerate}
    \item{Retourner le thermomètre à gaz pour que la partie scellée se retrouve en bas}
    \item{En controlant la pression avec la vanne du thermomètre à gaz, réduire la dépression $\Delta p$ jusqu'à 0}
\end{enumerate}

\begin{figure}[!h]
    \centering
    \caption{Photo Thermomètre à gaz, manomètre et pompe à main}
%    \includegraphics[angle=269, totalheight=21em]{Data/mano.jpg}
\end{figure}

\begin{itemize}
    \item{$T_1 = 280K = 6.85C\degree$}
    \item{$T_2 = T_{ambiante}$}
    \item{$T_3 = 340K = 66.85C\degree$}
\end{itemize}

\begin{align*}
    &T_{ini} = 7.5C\degree\\
    &T_{fin} = 11C\degree
\end{align*}

\begin{equation}
    \frac{3}{8}\cdot x + 7.5
\end{equation}

\begin{table}[!h]
    \centering
    \caption{Résultats T=280K}
    \begin{tabular}{|l|l|l|}
	\hline
	$\Delta p [Pa]$	&Hauteur [cm] &Temperature [C\degree]\\
	\hline
	100	&10.3 & 7.50  \\
	200	&11.5 & 7.88  \\
	300	&13.1 & 8.25  \\
	400	&15.3 & 8.63  \\
	500	&17.7 & 9.00  \\
	600	&21.1 & 9.38  \\
	650	&23.7 & 9.75  \\
	700	&26.6 & 10.13 \\
	720	&28.2 & 10.50 \\
	740	&29.2 & 10.88 \\
	760	&32.1 & 11.25 \\
	780	&33.6 & 11.63 \\
	\hline
    \end{tabular}
\end{table}
