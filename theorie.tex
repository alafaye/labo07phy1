\subsection{Chaleur massique}
Lors ce que l'on chauffe ou on refroidit un corps, l'énergie absorbée sous forme de chaleur $\Delta Q$ est proportionnelle au changement de température $\Delta \theta$ et la masse m du corps:
\begin{equation}
    \Delta Q = m\cdot c \cdot \Delta \theta
\end{equation}

Où c représente la chaleur massique, un coefficient dépendant de la nature du matériau. Pour le déterminer, 

Quantité de chaleur cédée par la grenaille:
\begin{equation}
    \label{deltaq}
    \Delta Q_1 = m_1 \cdot c_1 \cdot (\theta_1 -\theta_M)
\end{equation}
qui est égale à celle absorbée par l'eau.
\begin{equation}
    \Delta Q_2 = m_2 \cdot c_2 \cdot (\theta_M -\theta_2)
\end{equation}
On suppose ici qu'on connaît déjà la valeur de la chaleur massique de l'eau $c_2$, et $\theta_1 = \sim 98C\degree$ est égal à la température de la vapeur d'eau.
En utilisant les deux formules précédentes:
\begin{equation}
    c_1 = c_2\cdot \frac{m_2\cdot (\theta_M - \theta_2)}{m_1 \cdot (\theta_1 - \theta_M)}
\end{equation}

Il faut aussi prendre en compte que le vase calorimétrique dans lequel le mélange va être fait. La quantité de chaleur qu'il va absorber peut être donnée en masse équivalente d'eau: $m_{cal}=0.023kg$.
\begin{equation}
    \Delta Q_2 = (m_2+m_{cal}) \cdot c_2 \cdot (\theta_M -\theta_2)
\end{equation}
Ce qui donne donc:
\begin{equation}
    \label{final_cm}
    c_1 = c_2\cdot \frac{(m_2 + m_{cal})\cdot(\theta_M - \theta_2)}{m_1 \cdot (\theta_1 - \theta_M)}
\end{equation}

\subsection{Chaleur latente de vaporisation}
Quand un corps plongé dans un milieu où la pression est fixe est réchauffé, sa température augmente. Sauf si la substance du corps est dans une transition entre deux phases, toute l'énérgie apportée est alors dissipée pour passer d'une phase à l'autre. Dès que le chagement est terminé, la température va de nouveau augmenter.
Une bonne manière d'observer ce phénomène est en 

La chaleur est donc partagée entre la partie absorbée par~\eqref{deltaq}.

\begin{equation}
    \Delta Q_2 = m_1 \cdot L_v
\end{equation}


\begin{equation}
    \Delta Q_3 = m_2 \cdot c_e \cdot (\theta_M - \theta_2)
\end{equation}

\begin{equation}
    \Delta Q_4 = m_{cal} \cdot c_e \cdot (\theta_M - \theta_2)
\end{equation}

$\Delta Q_3 + \Delta Q_4$

\begin{equation}
    m_1 \cdot c_e \cdot (\theta_V - \theta_M) + m_1 \cdot L_V = (m_2 + m_{cal}) \cdot c_e \cdot (\theta \theta)
\end{equation}

\begin{equation}
    \label{final_lv}
    L_V = c_e \cdot \frac{(m_2 + m_{cal})\cdot (\theta_M - \theta_2) - m_1 \cdot (\theta_V - \theta_M)}{m_1}
\end{equation}

\subsection{Chaleur latente de fusion}
\begin{equation}
    \Delta Q_1 = m_1 \cdot c_e \cdot (\theta_M - \theta_1)
\end{equation}

\begin{equation}
    \Delta Q_2=m_1 \cdot L_F
\end{equation}

\begin{equation}
    \Delta Q_3 = m_2 \cdot c_e \cdot (\theta_2 - \theta_M)
\end{equation}

\begin{equation}
    \Delta Q_4 = m_{cal} \cdot c_e \cdot (\theta_2 - \theta_M)
\end{equation}

\begin{equation}
    \label{final_lf}
    L_F = c_e \cdot \frac{(m_2 + m_{cal}) \cdot (\theta_2 -\theta_M)-m_1 \cdot (\theta_M - \theta_1)}{m_1}
\end{equation}

\subsection{Incertitudes}

Une mesure expérimentale est toujours accompagnée de son $\textit{incertitude de mesure}$. On peut qualifier cette incertitude selon différentes caractéristiques:
\begin{itemize}
\item La résolution
\item La précision
\item La reproductibilité
\end{itemize}

Cette incertitude a de multiples sources, humaines ou liées au matériel, qui la rendent inévitable mais pas pour autant non-quantifiable.
On la calcule en générale de deux manières, sous forme $\textit{d'incertitude absolue}$ ou $\textit{d'incertitude relative}$.

\paragraph{Notation}
Un résultat s'écrit donc sous la forme: \[a\pm\Delta a\] tel que $\Delta a$ soit l'incertitude absolue et $\frac{\Delta a}{a} \%$ soit l'incertitude relative.

\paragraph{Chiffres significatifs}
Pour noter correctement les résultats avec leur incertitude, il faut être attentif aux $\textit{chiffres significatifs}$, les mesures et résultats de calculs doivent être exprimés avec un ou x chiffres dont la valeur n'est pas certaine.\\
Par exemple, pour une mesure au gramme près $m = (2.3\pm0.1)kg$.

\paragraph{Propagation des erreurs}
Une fois l'incertitude estimée, il faut encore la propager à tous les calculs qui suivront. Dans le cas d'une fonction à une seule variable, on peut utiliser sa différentielle:
\begin{equation}
    dg=|{\frac{df}{dx}}|dx
\end{equation}

Ce qui donne pratiquement:
\begin{equation}
    \Delta g=|\frac{df}{dx}|\Delta x
\end{equation}

Pour propager les erreurs durant les calculs, les formules suivantes sont utilisées:\\
Addition/Soustraction:
\begin{equation}
    \begin{split}
	g = a \pm b\\
	\Delta g = \Delta a + \Delta b
    \end{split}
\end{equation}
Multiplication/Division:
\begin{equation}
    \label{multdiv}
    \begin{split}
	g = k\cdot ab \;\;\; k=constante\\
	\frac{\Delta g}{|g|} = \frac{\Delta a}{|a|} + \frac{\Delta b}{|b|}
    \end{split}
\end{equation}
